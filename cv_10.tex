%%%%%%%%%%%%%%%%%%%%%%%%%%%%%%%%%%%%%%%%%
% Friggeri Resume/CV
% XeLaTeX Template
% Version 1.0 (5/5/13)
%
% This template has been downloaded from:
% http://www.LaTeXTemplates.com
%
% Original author:
% Adrien Friggeri (adrien@friggeri.net)
% https://github.com/afriggeri/CV
%
% License:
% CC BY-NC-SA 3.0 (http://creativecommons.org/licenses/by-nc-sa/3.0/)
%
% Important notes:
% This template needs to be compiled with XeLaTeX and the bibliography, if used,
% needs to be compiled with biber rather than bibtex.
%
%%%%%%%%%%%%%%%%%%%%%%%%%%%%%%%%%%%%%%%%%

\documentclass[]{friggeri-cv} % Add 'print' as an option into the square bracket to remove colors from this template for printing

\addbibresource{bibliography.bib} % Specify the bibliography file to include publications

\begin{document}

\header{Dr. Philipp}{ Bayer}{Postdoctoral researcher} % Your name and current job title/field

%----------------------------------------------------------------------------------------
%	SIDEBAR SECTION
%----------------------------------------------------------------------------------------

\begin{aside} % In the aside, each new line forces a line break
\section{contact}
\href{mailto:philippbay@gmail.com}{philippbay@gmail.com}
\href{http://github.com/philippbayer}{github.com/philippbayer}
\href{http://twitter.com/philippbayer}{twitter.com/philippbayer}
\section{languages}
German: mother tongue
English: fluent
French \& Japanese: advanced
\section{programming}
Python + R
Go, Perl, Bash, Java
Ruby on Rails, JavaScript, HTML
\section{research}
Genomics of complex traits in plants
\section{statistics}
Citations: 4,027
h-index: 24
i10-index: 37
\end{aside}

%----------------------------------------------------------------------------------------
%	EDUCATION SECTION
%----------------------------------------------------------------------------------------

\section{education}

\begin{entrylist}
%------------------------------------------------
\entry
{2012--2015}
{PhD {\normalfont Applied Bioinformatics}}
{University of Queensland, Brisbane}
{Working in the Applied Bioinformatics group on the use of genotyping by sequencing to improve the genome assembly of canola.\\Thesis submission date: 23rd September 2015. Date of acceptance: 4th May 2016.}
%------------------------------------------------
\entry
{2010--2012}
{Master {\normalfont of IT}}
{Bond University, Gold Coast}
{Graduated with High Distinction}
%------------------------------------------------
\entry
{2006--2009}
{Bachelor of Science {\normalfont Biology}}
{University of Münster, Germany}
{Thesis: Analysis of splicing in two populations of marine plants
using bioinformatic approaches}
\end{entrylist}

\section{employment}
\begin{entrylist}
\entry
{2020-2023}
{DECRA Fellow}
{UWA, Perth}
{My first step towards an independent lab with funding for the first PhD-student primarily supervised by me and with \$448,365 in government and \$418,772 UWA funding.}
\entry
{2017-2020}
{Forrest Fellow}
{UWA, Perth}
{Edwards Lab. Continued work on genomics of complex plants with Forrest Foundation support. Supervised three PhD students and four MSc students to completion.}
\entry
{2015--2017}
{Postdoctoral researcher}
{UWA, Perth}
{Edwards Lab. Researched genetics of complex plants with a focus on canola and wheat. Worked closely with industry partners to improve their breeding programs. Preparing, writing, and publishing research. Supervised two interns, Co-supervised four PhD students and one MSc student, system-administrator for the local computational infrastructure and group data manager. Worked extensively on an ARC Industrial Transformation Training Centre application (2018 round).}
\end{entrylist}
%------------------------------------------------

%----------------------------------------------------------------------------------------
%	PUBLICATIONS SECTION
%----------------------------------------------------------------------------------------

\section{publications}
\cite{yang2021candidate}
\cite{yang2021genome}
\cite{valliyodan2021genetic}
\cite{danilevicz2021high}
\cite{li2020assembly}
\cite{bayer2020machine}
\cite{merce2020induced}
\cite{tirnaz2020resistance}
\cite{bayer2020plant}
\cite{tirnaz2020effect}
\cite{inturrisi2020genome}
\cite{tirnaz2020resistance}
\cite{dolatabadian2020characterization}
\cite{golicz2020pangenomics}
\cite{anderson2020method}
\vskip 0.3\baselineskip  % have to add these for book chapters - TODO add to friggeri.cls
\cite{hu2020legume}
\cite{danilevicz2020plant}
\cite{zhao2020trait}
\cite{anderson2020climate}
\cite{yuan2020refka}
\cite{furaste2020plant}
\cite{valliyodan2019construction}
\cite{kreplak2019reference}
\cite{Dolatabadian_2019}
\cite{Tahghighi2019}
\cite{Mousavi2019}
\cite{Kreplak2019}
\cite{Valliyodan2019}
\cite{scheben2019cropsnpdb}
\cite{anderson2019}
\cite{mousavi2019adapting}
\cite{taylor2019indel}
\cite{melonek2019high}
\cite{bayer2019variation}
\cite{bayer2018bias}
\cite{mousavi2018western}
\cite{the_international_wheat_genome_sequencing_consortium_iwgsc_shifting_2018}
\cite{ramirez-gonzalez_transcriptional_2018}
\cite{lee2018genomic}
\cite{mousavi2018exploring}
\cite{hurgobin2018homoeologous}
\cite{schneider2017establishing}
\cite{kaur2017climate}
\cite{yuan2017runbng}
\cite{yuan2017bionanoanalyst}
\cite{bayer2017assembly}
\cite{yuan2017improvements}
\cite{montenegro2017pangenome}
\cite{kaur2017advanced}
\cite{gacek2017genome}
\cite{golicz2016pangenome}
\cite{hane2017comprehensive}
\cite{barash2016candidate}
\cite{lee2016genome}
\cite{bayer2016genomics}
\vskip 0.3\baselineskip  % have to add these for book chapters - TODO add to friggeri.cls
\cite{visendi2016efficient}
\cite{mason2016centromere}
\cite{bayer2016skim}
\vskip 0.3\baselineskip 
\cite{bayer2015high}
\cite{golicz2015skim}
\vskip 0.3\baselineskip 
\cite{lai2015identification}
\cite{chalhoub2014early}
\cite{mason2014high}
\cite{greshake2014opensnp}
\cite{dattolo2013acclimation}

%printbibsection{article}{articles in peer-reviewed journals} % Print all articles from the bibliography

%\printbibsection{book}{books} % Print all books from the bibliography

%\begin{refsection} % This is a custom heading for those references marked as "inproceedings" but not containing "keyword=france"
%\nocite{*}
%\printbibliography[sorting=chronological, type=inproceedings, title={international peer-reviewed conferences/proceedings}, notkeyword={france}, heading=subbibliography]
%\end{refsection}

%\begin{refsection} % This is a custom heading for those references marked as "inproceedings" and containing "keyword=france"
%\nocite{*}
%\printbibliography[sorting=chronological, type=inproceedings, title={local peer-reviewed conferences/proceedings}, keyword={france}, heading=subbibliography]
%\end{refsection}

%----------------------------------------------------------------------------------------


\newpage
%----------------------------------------------------------------------------------------
%	WORK EXPERIENCE SECTION
%----------------------------------------------------------------------------------------
\section{experience}

\begin{entrylist}
\entry
{2021--Current}
{Member, Scientific Advisory Panel Machine Learning}
{ARDC}
{Member of the scientific advisory panel for ongoing machine learning projects supported by the ARDC.}
%------------------------------------------------
\entry
{2012--2018}
{Research collaboration with Bayer CropScience, later BASF}
{Ghent, Belgium}
{Continued collaboration with Bayer CropScience on their plant breeding projects which includes several week-long visits to Bayer.}
%------------------------------------------------
\entry
{2011--Current}
{Co-founder openSNP.org}
{Germany/Australia}
{A project for customers of genotyping companies like 23andMe to share their data with scientists around the world, for free. Partially wrote and still maintain the site's Ruby on Rails code-base, interact and manage with the community of 5000 users, administration of the site's servers, and supervision of contributors.}
%------------------------------------------------
\entry
{2013--Current}
{Software Carpentry and Data Carpentry instructor}
{Australia}
{Certified Software Carpentry and Data Carpentry instructor. Software Carpentry is a Mozilla/Alfred P. Sloan Foundation funded non-profit organization which teaches best programming practices (structured programming, reproducible research, version control etc.) in bootcamps to scientists around the world. Data Carpentry is a sister-organisation that focuses on teaching best data management practices.}
%------------------------------------------------
\entry
{2018}
{Research Bazaar Organising committee}
{UWA, Perth}
{ResBaz is a world-wide three-day festival promoting digital literacy. As member of the organising committee I searched for helpers and teachers, drafted the timeplan, designed the web page, raised funding, succeeded in getting government MP to hold keynote}
%------------------------------------------------
\entry
{2017--Current}
{Hacky Hour Founder}
{UWA, Perth}
{Founded the Hacky Hour at UWA, a weekly get-together of researchers and staff working with programming and data, doubles as a help-desk for students with programming problems.}
%------------------------------------------------
\entry
{2017--2019}
{Mozilla Open Science Leadership mentor}
{UWA, Perth}
{Mentoring open source programmers and researchers on how to streamline and grow open source and open science projects under the umbrella of Mozilla.}
%------------------------------------------------
\entry
{2016--2019}
{EMBL-ABR Head of Nodes member, Open Science Special Interest Group member}
{UWA, Perth}
{EMBL-ABR is an Australian-wide network supporting the technical needs of life sciences researchers. Members of the group of Head of Nodes meet monthly to discuss the way forward for the organisation. The Open Science Special Interest Group meets bimonthly to discuss how EMBL-ABR can advance open science in Australia.}
%------------------------------------------------
\entry
{2016--2017}
{COMBINE WA Representative}
{UWA, Perth}
{COMBINE is the student and early career researcher subcommittee of the Australian Bioinformatics and Computational Biology Society (ABACBS). As the local representative I organise or help organise workshops and regular networking events.}
%------------------------------------------------
\end{entrylist}

\newpage
%----------------------------------------------------------------------------------------
%	AWARDS SECTION
%----------------------------------------------------------------------------------------

\section{awards \& funding}

\begin{entrylist}
%------------------------------------------------
\entry
{2021}
{ARC Discovery Early Career Research Award}
{UWA}
{Awarded DECRA for 2021-2023.}
%------------------------------------------------
\entry
{2019}
{Woodside Early Career Scientist of the Year, finalist}
{UWA}
{Finalist in Premier's Science Awards 2019}
%------------------------------------------------
\entry
{2018}
{Rising Stars nomination}
{UWA}
{Two early career researchers per UWA research school were nominated for Rising Stars, a university-wide event where researchers introduce a public audience to their research}
%------------------------------------------------
\entry
{2018}
{Forrest Research Foundation Non-stipendiary Fellowship}
{UWA}
{Three year fellowship to pursue research at UWA, part of the three inaugural Forrest Fellows}
%------------------------------------------------
\entry
{2017}
{UWA Research Collaboration Award}
{UWA}
{\$28,100 to fund a seagrass microbiome sequencing project}
%------------------------------------------------
\entry
{2014}
{GRDC Travel Award}
{GRDC}
{Travel cost scholarship}
%------------------------------------------------
\entry
{2014}
{SAFS Travel Award}
{University of Queensland}
{Travel cost scholarship}
%------------------------------------------------
\entry
{2013}
{Bayer Grants4Apps}
{Bayer HealthCare}
{Grant to cover openSNP running costs}
%------------------------------------------------
\entry
{2011--2014}
{Two postgraduate scholarships}
{University of Queensland}
{My PhD was supported by two scholarships from UQ for tuition fees and living costs.}
%------------------------------------------------
\entry
{2012}
{First place in PLOS/Mendeley Binary Challenge}
{Won with openSNP.org}
{Won first price in a competition aimed towards the advancement of open science}
%------------------------------------------------
\entry
{2009-2011}
{Master IT}
{Bond University}
{5x Top of class,  3x Vice-Chancellor's List for Academic Excellence, 1x IT Award Academic Excellence. Graduated with High Distinction. Recipient of John Oglethorpe Medal for highest GPA of all IT students graduating that semester}
\end{entrylist}


%----------------------------------------------------------------------------------------
%	TEACHING SECTION
%----------------------------------------------------------------------------------------
\section{teaching}

\begin{entrylist}
\entry
{2019}
{Introduction to genomics on the command line}
{Research Bazaar, Curtin University, Perth}
{Introduction to the command line, bioinformatics analyses and pipelines, and basic SNP analysis in R}
\entry
{2019}
{Introduction to tidyverse and caret in R}
{UWA School of Human Sciences, Perth}
{Introduction to R, tidyverse, ggplot2, caret, and basic statistics in R. Taught over two days.}
\entry
{2018}
{Introduction to modern R}
{Telethon Kids Institute, Perth}
{Introduction to R, tidyverse, ggplot2, and basic statistics approaches in R. Taught over two days.}
%------------------------------------------------
\entry
{2018}
{Teaching Data Carpentry}
{Research Bazaar, University of Western Australia, Perth}
{Introduction to genomics and shell. Part of the planning committee.}
%------------------------------------------------
\entry
{2017}
{Teaching Software Carpentry}
{Research Bazaar, Curtin University, Perth}
{Introduction to data manipulation using Python}
%------------------------------------------------
\entry
{2016}
{Teaching and hosting Data Carpentry}
{UWA, Perth}
{Hosted, planned, and set up the first Data Carpentry workshop at UWA, taught best data management practices}
%------------------------------------------------
\end{entrylist}
\begin{entrylist}
\entry
{2021--Current}
{University teaching}
{UWA, Perth}
{Organised new MSc Bioinformatics with new unit, SCIE5003 (advanced bioinformatics). Developed content of SCIE5003 and SCIE4002, taught into both units.}
%------------------------------------------------
\entry
{2016--Current}
{University teaching}
{UWA, Perth}
{Co-teach and co-supervise SCIE4002, computational analysis for biology and biomedical MSc students. Set up and maintain the computational infrastructure needed for practicals. In 2017, the course has been judged 'consistenly excellent' over six semesters based on student evaluations.}
%------------------------------------------------
\entry
{2016}
{Teaching Software Carpentry}
{Curtin University, Perth}
{Taught introduction to Python}
%------------------------------------------------
\entry
{2016}
{Teaching Software Carpentry}
{Research Bazaar, Murdoch University, Perth}
{Taught introduction to Python and git}
%------------------------------------------------
\entry
{2016}
{Teaching and hosting Software Carpentry}
{University of Queensland, Brisbane}
{Hosted, planned, and set up the first Software Carpentry workshop at UQ. Taught introduction to programming.}
%------------------------------------------------
\entry
{2014}
{Teaching Software Carpentry}
{Sydney}
{Taught basic to intermediate Python.}
%------------------------------------------------
\entry
{2014}
{Teaching Software Carpentry}
{PyCon AU/University of Queensland}
{Taught basic to intermediate Python.}
%------------------------------------------------
\entry
{2013}
{Teaching Software Carpentry}
{Adelaide}
{Assisted Software Carpentry bootcamp in Adelaide}
%------------------------------------------------
\entry
{2009--2011}
{Tutoring}
{Bond University}
{Tutored students in Intro to Programming (Java), Database Management (Oracle/MySQL) and Networks \& Applications, held several all-day refresher courses before exams}
%------------------------------------------------
\end{entrylist}


\section{presentations}
\begin{entrylist}
%------------------------------------------------

\entry
{2021}
{Machine learning in plant breeding and bioinformatics: what comes next}
{}
{CINVESTAV, online}
\entry
{2021}
{Future-ready crops for a changing climate: the role of bioinformatics}
{}
{UWA DVCR Forrest Foundation seminar series}
%------------------------------------------------
\entry
{2021}
{Bioinformatics at scale panel Q\&A}
{}
{Pawsey Supercomputing Centre}
%------------------------------------------------
\entry
{2021}
{Interpretable machine learning in bioinformatics}
{}
{ABACBS online seminar series}
%------------------------------------------------
\entry
{2020}
{Our machine learning technical stack}
{}
{GRDC Tech seminar series}
%------------------------------------------------
\entry
{2020}
{Predicting Gene Loss in Plants: Lessons Learned from Laptop-Scale Data}
{}
{Plant And Animal Genome conference, San Diego}
%------------------------------------------------
\entry
{2019}
{Eukaryotic pangenomics: where we've been, where we're going}
{}
{Bayliss Seminar Series, Perth}
%------------------------------------------------
\entry
{2019}
{Assembling complex plant genomes – things I wish someone would have told me earlier}
{}
{AGRF Seminar Series, Perth}
%------------------------------------------------
\entry
{2019}
{Helping Biologists Make Sense of Plant Variant and Annotation Data}
{}
{Plant And Animal Genome conference, San Diego}
%------------------------------------------------
\entry
{2018}
{Feeding the future world: safe-guarding Australia’s food bowl in a changing climate}
{}
{Rising Stars, UWA}
%------------------------------------------------
\end{entrylist}
\begin{entrylist}
\entry
{2018}
{From QTLs to candidate genes, or: There and Back Again}
{}
{Institute of Agriculture Seminar Series, UWA}
%------------------------------------------------
\entry
{2018}
{The path of least resistance (genes) - mining plant genomes for disease resistance}
{}
{COMBINE/Pawsey bioinformatics symposium}
%------------------------------------------------
\entry
{2018}
{Early Career Researcher Panel - What have I learnt at the beginning of my research career?}
{}
{Combined Biological Sciences Meeting 2018}
%------------------------------------------------
\entry
{2018}
{ScienceCafe - STEM outreach aimed at year 10 students}
{}
{UWA}
%------------------------------------------------
\entry
{2017}
{The future of wheat research}
{}
{Wheat showcase, UWA}
%------------------------------------------------
\entry
{2017}
{Skipping the assembly step – what we can learn from looking at sequences directly}
{}
{Pawsey Roadshow, UWA, Perth}
%------------------------------------------------
\entry
{2017}
{The State of Bioinformatics in High Performance Computing in 2017}
{}
{HPCAC Conference, Perth}
%------------------------------------------------
\entry
{2017}
{Towards better plant breeding at UWA}
{}
{COMBINE event, Perth}
%------------------------------------------------
\entry
{2017}
{Improving Plant Breeding using KNetMiner}
{}
{Plant And Animal Genome conference, San Diego}
%------------------------------------------------
\entry
{2016}
{Towards a canola pan-genome: cautionary tales from the assembly bench}
{}
{CCDM, Curtin University}
%------------------------------------------------
\entry
{2016}
{Sharing Experience: What Can We Learn from Each Other Developing Plant Informatics Systems}
{}
{Plant And Animal Genome conference, San Diego}
%------------------------------------------------
\entry
{2015}
{Assessing and validating the amphidiploid genome of \textit{Brassica napus} using genotyping by sequencing}
{}
{Plant And Animal Genome conference, San Diego}
%------------------------------------------------
\entry
{2015}
{Using skim-based genotyping by sequencing for trait association and QTL cloning in Brassica napus}
{}
{Plant And Animal Genome conference, San Diego}
%------------------------------------------------
\entry
{2014}
{Assembling and validating the genome of the \textit{Brassica napus} using skim-based genotyping by sequencing}
{}
{University of Queensland, GenGen Seminar Series}
%------------------------------------------------
\entry
{2012}
{openSNP: Crowdsourcing Genome Wide Association Studies}
{}
{28th Chaos Communication Congress, Berlin}
%------------------------------------------------
\end{entrylist}
\end{document}
