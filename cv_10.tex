%%%%%%%%%%%%%%%%%%%%%%%%%%%%%%%%%%%%%%%%%
% Friggeri Resume/CV
% XeLaTeX Template
% Version 1.0 (5/5/13)
%
% This template has been downloaded from:
% http://www.LaTeXTemplates.com
%
% Original author:
% Adrien Friggeri (adrien@friggeri.net)
% https://github.com/afriggeri/CV
%
% License:
% CC BY-NC-SA 3.0 (http://creativecommons.org/licenses/by-nc-sa/3.0/)
%
% Important notes:
% This template needs to be compiled with XeLaTeX and the bibliography, if used,
% needs to be compiled with biber rather than bibtex.
%
%%%%%%%%%%%%%%%%%%%%%%%%%%%%%%%%%%%%%%%%%

\documentclass[]{friggeri-cv} % Add 'print' as an option into the square bracket to remove colors from this template for printing

\addbibresource{bibliography.bib} % Specify the bibliography file to include publications

\begin{document}

\header{philipp}{bayer}{bioinformatician} % Your name and current job title/field

%----------------------------------------------------------------------------------------
%	SIDEBAR SECTION
%----------------------------------------------------------------------------------------

\begin{aside} % In the aside, each new line forces a line break
\section{contact}
\href{mailto:philippbay@gmail.com}{philippbay@gmail.com}
\href{http://github.com/philippbayer}{github.com/philippbayer}
\section{languages}
German: mother tongue
English: fluent
French \& Japanese: advanced
\section{programming}
Python
Go, Perl, Bash, Java
Ruby on Rails, HTML
\end{aside}

%----------------------------------------------------------------------------------------
%	EDUCATION SECTION
%----------------------------------------------------------------------------------------

\section{education}

\begin{entrylist}
%------------------------------------------------
\entry
{2012--12/2014}
{PhD {\normalfont Applied Bioinformatics}}
{University of Queensland, Brisbane}
{Working in the applied bioinformatics group on the use of genotyping by sequencing to improve the genome assembly of canola}
%------------------------------------------------
\entry
{2010--2012}
{Master {\normalfont of IT}}
{Bond University, Gold Coast}
{Graduated with Honours}
%------------------------------------------------
\entry
{2006--2009}
{Bachelor of Science {\normalfont Biology}}
{University of Münster, Germany}
{Thesis: Analysis of splicing in two populations of marine plants
using bioinformatic approaches}
\end{entrylist}

%----------------------------------------------------------------------------------------
%	WORK EXPERIENCE SECTION
%----------------------------------------------------------------------------------------

\section{experience}

\begin{entrylist}
\entry
{2013--Current}
{Software Carpentry instructor}
{Australia}
{Certified Software Carpentry instructor. Software Carpentry is a Mozilla/Alfred P. Sloan Foundation funded non-profit organization which teaches best programming practices (structured programming, reproducible research, version control etc.) in bootcamps to scientists around the world}
%------------------------------------------------
\entry
{2012--2012}
{Research exchange {\normalfont at Bayer CropScience}}
{Ghent, Belgium}
{For 4 weeks, worked on the assembly of the \textit{Brassica napus} genome. Learned to work in a corporate science environment}

%------------------------------------------------
\entry
{2011--Now}
{Co-founder openSNP.org}
{Germany/Australia}
{A project for costumers of genotyping companies like 23andMe to share their data with scientists around the world, for free.\\
Responsibilities:
\begin{itemize}
\item Partially wrote and still maintain the site's Ruby on Rails code-base
\item Interact with the community of 1500 users
\item System administration of the site's server
\end{itemize}}

%------------------------------------------------

%------------------------------------------------
\end{entrylist}

%----------------------------------------------------------------------------------------
%	AWARDS SECTION
%----------------------------------------------------------------------------------------

\section{awards}

\begin{entrylist}
%------------------------------------------------
\entry
{2014}
{SAFS Travel Award}
{University of Queensland}
{\$2500 travel cost scholarship}
%------------------------------------------------
\entry
{2011--2014}
{Two postgraduate scholarships}
{University of Queensland}
{For the work on genotyping by sequencing, covers tuition and living costs.}
%------------------------------------------------
\entry
{2012}
{First place in PLOS/Mendeley Binary Challenge}
{Won with openSNP.org}
{Won first price in a competition aimed towards the advancement of open science}
%------------------------------------------------
\entry
{2009-2011}
{Master IT}
{Bond University}
{5x Top of class,  3x Vice-Chancellor's List for Academic Excellence, 1x IT Award Academic Excellence. Graduated with honours. Recipient of John Oglethorpe Medal for highest GPA of all IT students graduating that semester}
\end{entrylist}


%----------------------------------------------------------------------------------------
%	COMMUNICATION SKILLS SECTION
%----------------------------------------------------------------------------------------
\newpage
\section{communication skills}

\begin{entrylist}
%------------------------------------------------
\entry
{2015}
{Presentation}
{Plant and animal genome conference, San Diego}
{Assessing and validating the amphidiploid genome of \textit{Brassica napus} using genotyping by sequencing}
%------------------------------------------------
\entry
{2014}
{Presentation}
{University of Queensland, GenGen Seminar Series}
{Assembling and validating the genome of the \textit{Brassica napus} using skim-based genotyping by sequencing}
%------------------------------------------------
\entry
{2013}
{Poster}
{Plant and animal genome conference, San Diego}
{Genome Assembly Validation and Trait Association using Skim Based Genotyping by Sequencing in Canola}
%------------------------------------------------
\entry
{2013--2014}
{Software Carpentry}
{Adelaide/Melbourne}
{Assisted Software Carpentry bootcamp in Adelaide, taught basic to intermediate Python as well as documentation and assisted at bootcamp in Melbourne. Currently organizing a bootcamp in Brisbane in July 2014.}
%------------------------------------------------
%\entry
%{2012}
%{Presentation}
%{28th Chaos Communication Congress, Berlin}
%{Presented the work on openSNP, talked about the future of personal genomics and privacy implications}
%------------------------------------------------
\entry
{2009--2011}
{Tutoring}
{Bond University}
{Tutored students in Intro to Programming (Java), Database Management (Oracle/MySQL) and Networks \& Applications, held several all-day refresher courses before exams}
\end{entrylist}

%----------------------------------------------------------------------------------------
%	INTERESTS SECTION
%----------------------------------------------------------------------------------------

\section{interests}

\textbf{professional:} genotyping by sequencing, genome refinement, programming, machine learning, teaching \\ \textbf{personal:} literature, running, travel

%----------------------------------------------------------------------------------------
%	PUBLICATIONS SECTION
%----------------------------------------------------------------------------------------

\section{publications}

\printbibsection{article}{articles in peer-reviewed journals} % Print all articles from the bibliography

%\printbibsection{book}{books} % Print all books from the bibliography

%\begin{refsection} % This is a custom heading for those references marked as "inproceedings" but not containing "keyword=france"
%\nocite{*}
%\printbibliography[sorting=chronological, type=inproceedings, title={international peer-reviewed conferences/proceedings}, notkeyword={france}, heading=subbibliography]
%\end{refsection}

%\begin{refsection} % This is a custom heading for those references marked as "inproceedings" and containing "keyword=france"
%\nocite{*}
%\printbibliography[sorting=chronological, type=inproceedings, title={local peer-reviewed conferences/proceedings}, keyword={france}, heading=subbibliography]
%\end{refsection}

%----------------------------------------------------------------------------------------

\end{document}
